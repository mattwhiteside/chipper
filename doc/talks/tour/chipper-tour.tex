\documentclass[xcolor=pdflatex,dvipsnames,table]{beamer}
% \usepackage{pgfpages}
% \setbeameroption{show notes on second screen=left}
% \setbeamertemplate{note page}[plain]
% \setbeameroption{show notes}
\usepackage{epsfig,graphicx}
\usepackage{palatino}
\usepackage{fancybox}
\usepackage{relsize}
\usepackage[procnames]{listings}

% "define" Stanza
\usepackage[T1]{fontenc}  
\usepackage[scaled=0.82]{beramono}  
\usepackage{microtype} 

\sbox0{\small\ttfamily A}
\edef\mybasewidth{\the\wd0 }

\lstdefinelanguage{stanza}{
  morekeywords={circuit, defclass,defmodule,defbundle,definterface,defpackage,defn,%
    do,else,public,false,finally,%
    for,if,import,inherit,inst,match,%
    map,new,node,object,override,package,%
    super,this,throw,true,try,%
    type,val,var,when,while%,
    % yield,UInt,Bool,Bits,SInt
    },
  sensitive=true,
  morecomment=[l]{;;},
  morestring=[b]",
  morestring=[b]',
  morestring=[b]"""
}

\usepackage{color}
\definecolor{dkgreen}{rgb}{0,0.6,0}
\definecolor{gray}{rgb}{0.5,0.5,0.5}
\definecolor{mauve}{rgb}{0.58,0,0.82}

% Default settings for code listings
\lstset{frame=tb,
  language=stanza,
  aboveskip=3mm,
  belowskip=3mm,
  showstringspaces=false,
  columns=fixed, % basewidth=\mybasewidth,
  basicstyle={\small\ttfamily},
  numbers=none,
  numberstyle=\footnotesize\color{gray},
  % identifierstyle=\color{red},
  keywordstyle=\color{blue},
  commentstyle=\color{dkgreen},
  stringstyle=\color{mauve},
  frame=single,
  breaklines=true,
  breakatwhitespace=true,
  procnamekeys={input,output,wire, mem, reg, node, defn, val, var, defclass, definterface, defbundle, defmodule, defpackage},
  procnamestyle=\ttfamily\color{red},
  tabsize=2
}

\lstnewenvironment{stanza}[1][]
{\lstset{language=stanza,#1}}
{}

% "define" Stanza
\usepackage[T1]{fontenc}  
\usepackage[scaled=0.82]{beramono}  
\usepackage{microtype} 

\sbox0{\small\ttfamily A}
\edef\mybasewidth{\the\wd0 }

\lstdefinelanguage{stanza}{
  morekeywords={circuit, defclass,defmodule,defbundle,definterface,defpackage,defn,%
    do,else,public,false,finally,%
    for,if,import,inherit,input,inst,match,%
    map,mem,new,node,object,override,output,package,%
    reg,super,this,throw,true,try,%
    type,val,var,when,while,wire%,%
    % yield,UInt,Bool,Bits,SInt
    },
  sensitive=true,
  morecomment=[l]{;;},
  morestring=[b]",
  morestring=[b]',
  morestring=[b]"""
}

\usepackage{color}
\definecolor{dkgreen}{rgb}{0,0.6,0}
\definecolor{gray}{rgb}{0.5,0.5,0.5}
\definecolor{mauve}{rgb}{0.58,0,0.82}

% Default settings for code listings
\lstset{frame=tb,
  language=stanza,
  aboveskip=3mm,
  belowskip=3mm,
  showstringspaces=false,
  columns=fixed, % basewidth=\mybasewidth,
  basicstyle={\small\ttfamily},
  numbers=none,
  numberstyle=\footnotesize\color{gray},
  % identifierstyle=\color{red},
  keywordstyle=\color{blue},
  commentstyle=\color{dkgreen},
  stringstyle=\color{mauve},
  frame=single,
  breaklines=true,
  breakatwhitespace=true,
  procnamekeys={input,output,wire, mem, reg, node, defn, val, var, defclass, definterface, defbundle, defmodule, defpackage},
  procnamestyle=\ttfamily\color{red},
  tabsize=2
}



% "define" Scala
\usepackage[T1]{fontenc}  
\usepackage[scaled=0.82]{beramono}  
\usepackage{microtype} 

\sbox0{\small\ttfamily A}
\edef\mybasewidth{\the\wd0 }

\lstdefinelanguage{scala}{
  morekeywords={abstract,case,catch,class,def,%
    do,else,extends,false,final,finally,%
    for,if,implicit,import,match,mixin,%
    new,null,object,override,package,%
    private,protected,requires,return,sealed,%
    super,this,throw,trait,true,try,%
    type,val,var,while,with,yield},
  sensitive=true,
  morecomment=[l]{//},
  morecomment=[n]{/*}{*/},
  morestring=[b]",
  morestring=[b]',
  morestring=[b]"""
}

\usepackage{color}
\definecolor{dkgreen}{rgb}{0,0.6,0}
\definecolor{gray}{rgb}{0.5,0.5,0.5}
\definecolor{mauve}{rgb}{0.58,0,0.82}

% Default settings for code listings
\lstset{frame=tb,
  language=scala,
  aboveskip=3mm,
  belowskip=3mm,
  showstringspaces=false,
  columns=fixed, % basewidth=\mybasewidth,
  basicstyle={\small\ttfamily},
  numbers=none,
  numberstyle=\footnotesize\color{gray},
  % identifierstyle=\color{red},
  keywordstyle=\color{blue},
  commentstyle=\color{dkgreen},
  stringstyle=\color{mauve},
  frame=single,
  breaklines=true,
  breakatwhitespace=true,
  procnamekeys={def, val, var, class, trait, object, extends},
  procnamestyle=\ttfamily\color{red},
  tabsize=2
}

\lstnewenvironment{scala}[1][]
{\lstset{language=scala,#1}}
{}
\lstnewenvironment{cpp}[1][]
{\lstset{language=C++,#1}}
{}
\lstnewenvironment{bash}[1][]
{\lstset{language=bash,#1}}
{}
\lstnewenvironment{verilog}[1][]
{\lstset{language=verilog,#1}}
{}


\lstset{basicstyle={\footnotesize\ttfamily}}

\usetheme[height=8mm]{Rochester}
\setbeamersize{text margin left=3mm} 
\setbeamersize{text margin right=3mm} 
\setbeamertemplate{navigation symbols}{}

\definecolor{Cobalt}{rgb}{0.25,0.125,0.70}
\definecolor{RedOrange}{rgb}{0.8,0.25,0.0}
% \definecolor{RedOrange}{rgb}{0.8,0.775,0.25}
\def\frametitledefaultcolor{Cobalt}
\def\frametitleproblemcolor{RedOrange}

\lstset{basicstyle={\footnotesize\ttfamily}}

\setbeamertemplate{frametitle}
{
\vskip-7mm
\textbf{\insertframetitle}\hfill\insertframenumber
}
\setbeamercolor{frametitle}{bg=\frametitledefaultcolor}

\newenvironment{sample}{\VerbatimEnvironment\begin{footnotesize}\begin{semiverbatim}}{\end{semiverbatim}\end{footnotesize}}

\newenvironment{FramedSemiVerb}%
{\begin{Sbox}\begin{minipage}{.94\textwidth}\begin{semiverbatim}}%
{\end{semiverbatim}\end{minipage}\end{Sbox}
\setlength{\fboxsep}{8pt}\fbox{\TheSbox}}

\newenvironment{FramedVerb}%
{\VerbatimEnvironment
\begin{Sbox}\begin{minipage}{.94\textwidth}\begin{Verbatim}}%
{\end{Verbatim}\end{minipage}\end{Sbox}
\setlength{\fboxsep}{8pt}\fbox{\TheSbox}}

% \newenvironment{sample}{\VerbatimEnvironment\begin{footnotesize}\begin{Verbatim}}{\end{Verbatim}\end{footnotesize}}
\newcommand{\code}[1]{\begin{footnotesize}{\tt #1}\end{footnotesize}}
\newcommand{\comment}[1]{{\color{Green}\it\smaller #1}}


\title{Chipper}
\author{Patrick Li + {\Large Jonathan Bachrach}}
\date{\today}
\institute[UC Berkeley]{EECS UC Berkeley}

\begin{document}

\begin{frame}
\titlepage
\end{frame}
\addtocounter{framenumber}{-1}

\begin{frame}[fragile]{Downloading Chipper}
For these steps, it will be assumed that Chipper is in the directory, \verb+mydir/chipper+, and that Stanza is installed in the directory, \verb+mydir/stanza+. \\[0.5cm]

0. First install Stanza rev.375 into \verb+mydir/stanza+ from the \url{lbstanza.org} website. \\[0.5cm]

1. Clone \url{https://github.com/ucb-bar/chipper.git} into \verb+mydir/chipper+ using:

\begin{bash}
git clone https://github.com/ucb-bar/chipper.git
\end{bash}  

\end{frame}

\begin{frame}[fragile]{Configuring the Installation}
2. Navigate to mydir/mychipper and type:

\begin{bash}
stanza -e src/chipper-syntax.stanza -o chipperc -flags OPTIMIZE
\end{bash}

\noindent
This will extend stanza with chipper macros and produce an extended compiler called chipperc. \\[0.5cm]
   
3. To create the run script for chipperc, type the following:

\begin{bash}
./chipperc -platform os-x -path /mydir/stanza -install chipper
\end{bash}

\noindent
This will create a small script for running chipperc with the above options. The path options tells chipperc where to find the system source files.
\end{frame}

\begin{frame}[fragile]{Testing the Installation}
4. Now to compile an example, type:

\begin{bash}
./chipper -i tests/gcd.stanza -o gcd
\end{bash}

\noindent
This compiles the chipper design gcd.stanza with the extended compiler and produces the binary gcd. \\[0.5cm]

5. Run the design by typing:

\begin{bash}
./gcd
\end{bash}
\end{frame}

\begin{frame}[fragile]{Chipper Examples}
\begin{itemize}
\item \verb+adder.stanza+ -- An implementation of an n-bit Adder. Demonstrates defining modules, and instances, wires, as well as generating hardware using a for loop.
\item \verb+alu.stanza+ -- An implementation of a simple ALU that performs addition and subtraction between two operands.
\item \verb+basicalu.stanza+ --  An implementation of a more involved ALU that performs, addition and some comparison operations.
\item \verb+gcd.stanza+ -- An implementation of a GCD circuit.
\item \verb+dualgcd.stanza+ -- An alternate implementation of the same GCD circuit above. It demonstrates using a user-defined parametric bundle type.
\item \verb+byteselector.stanza+ -- An implementation of a byte selector circuit. Demonstrates retrieving a subset of bits from a UInt.
\item \verb+life.stanza+ -- An implementation of Conway's Game of Life. 
\end{itemize}
\end{frame}

\begin{frame}[fragile]{Defining a Module}
The following defines a module that adds two numbers together. There are two inputs, called \verb+a+ and \verb+b+, each declared to be unsigned integers, and a single output, called \verb+z+, declared to be an 4-bit unsigned integers. Each port in the module is defined with the keywords \verb+input+, or \verb+output+. After the port listing comes the body of the module where all the connections are made. The \verb+MyAdder+ module's body has a single statement that connects the output \verb+z+ to the result of adding \verb+a+ to \verb+b+:

\begin{stanza}
defmodule MyAdder :
   input a: UInt<4>
   input b: UInt<4>
   output z: UInt<4>
   z := a + b
\end{stanza}
\end{frame}

\begin{frame}[fragile]{Parameterized Modules}
To parameterize a module, we can follow the module name with a list of arguments. The following example shows how to parameterize the width of the inputs and outputs of the \verb+MyAdder+ module:

\begin{stanza}
defmodule MyAdder (w:Int) :
   input a: UInt<w>
   input b: UInt<w>
   output z: UInt<w>
   z := a + b
\end{stanza}
\end{frame}

\begin{frame}[fragile]{Type Parameterized Modules}
To parameterize a module by type, we can declare the type parameters following the module name surrounded with angle brackets. The following example shows a one-cycle delay module parameterized by type:
\begin{stanza}
defmodule MyDelay<T> :
   input a: T
   output z: T

   reg r: T
   r := a
   z := r
\end{stanza}
\end{frame}

\begin{frame}[fragile]{Instantiating a Module}
To create an instance of a module we use the following syntax:
\begin{stanza}
inst i : MyAdder(10)
inst i : MyDelay<UInt<4>>
\end{stanza}

Notice that argument parameters and type parameters are differently. As a further example, the following module delays the inputs by a cycle before adding the result:
\begin{stanza}
defmodule DelayedAdder (w:Int) :
   input a: UInt<w>
   input b: UInt<w>
   output z: UInt<w>

   inst da : MyDelay<UInt<w>>
   inst db : MyDelay<UInt<w>>
   da.a := a
   db.a := b

   z := da.z + db.z
\end{stanza}
\end{frame}

\begin{frame}[fragile]{Chipper Types}
At the top-most level, we create a circuit using the following syntax:
\begin{stanza}
circuit c : MyAdder(10)
\end{stanza}
\noindent
This will create a circuit with the top-level module declared to \verb+MyAdder+. The value \verb+c+ will now contain the FIRRTL datastructure corresponding to the \verb+MyAdder+ circuit.
\end{frame}

\begin{frame}[fragile]{Chipper Types}
Types in chipper are used in the declaration of ports, instances, registers, and wires. Chipper provides a small number of built-in types, and supports user-defined bundle types:
\begin{stanza}
UInt               ;; unsigned integer type
UInt<4>            ;; unsigned integer type with a given bitwidth
UInt[10]           ;; vector of 10 unsigned integers
UInt<4>[10]        ;; vector of 10 unsigned integers, each of bitwidth 4
Bool               ;; one bit unsigned boolean
SInt               ;; signed integer type
T                  ;; type parameter declared in module/bundle param list
MyBundle           ;; user defined bundle type
MyBundle(10)       ;; user defined bundle type with value parameter argument
MyBundle<UInt>(10) ;; user defined bundle type with type and value parameter
\end{stanza}
\end{frame}

\begin{frame}[fragile]{User Defined Bundle Types}
Users may define their own bundle types using the following syntax:
\begin{stanza}
defbundle Cplx :
   real: UInt
   imag: UInt
\end{stanza}
The above is an example of a bundle type called \verb+Cplx+ with two fields, \verb+real+ and \verb+imag+, that are both declared to be unsigned integers of unknown bit-width.
\end{frame}

\begin{frame}[fragile]{User Defined Parameterized Bundle Types}
User-defined bundle type can be parameterized in the same way as modules, with both argument parameters and type parameters. The following code shows how to define a parameterized \verb+Decoupled+ type:
\begin{stanza}
defbundle Decoupled<T> :
   data: T
   valid: Bool
   flip ready: Bool
\end{stanza}
\noindent
Notice the keyword \verb+flip+ in the declaration of the \verb+ready+ field. This tells Chipper that the ready field allows connects in the opposite direction of the \verb+data+ and \verb+valid+ fields.
\end{frame}

\begin{frame}[fragile]{User Defined Functions on Bundle Types}
Users may define their own functions on bundle types by creating a standard Stanza function. The following Stanza function defines a \verb+plus+ function that allows adding two \verb+Cplx+ values:
\begin{stanza}
defn plus (a:Cplx, b:Cplx) -> Cplx :
   wire z: Cplx
   z.real := a.real + b.real
   z.imag := a.imag + b.imag
   z
\end{stanza}

\noindent
Using the above two definitions of the \verb+Cplx+ bundle type and the plus function, we may write an adder module that is able to add two complex numbers together.   

\begin{stanza}
defmodule CplxAdder :
   input a: Cplx
   input b: Cplx
   output z: Cplx
   z := plus(a, b)
\end{stanza}
\end{frame}

\begin{frame}[fragile]{User Defined Functions on Bundle Types}
In Stanza however, the syntax \verb+a + b+ is equivalent to a call to the \verb+plus+ function \verb+plus(a, b)+. So the above module can be written more legibly like this:

\begin{stanza}
defmodule CplxAdder :
   input a: Cplx
   input b: Cplx
   output z: Cplx
   z := a + b
\end{stanza}
\end{frame}

\begin{frame}[fragile]{Defining Wires and Registers}
Wires and Registers are defined using the keywords \verb+wire+, and \verb+reg+, respectively. Users must provide a name and a Chipper type and an optional value:
\begin{stanza}
wire w: UInt<1>     ;; wire with type
wire w = UInt(0)    ;; holding value
wire w := UInt(0)   ;; reassignable wire
\end{stanza}
\begin{stanza}
reg r: Cplx         ;; reg with type
reg r = Cplx(0, 1)  ;; reg + onreset value
reg r := Cplx(0, 1) ;; reg + next value
\end{stanza}

\end{frame}

\begin{frame}[fragile]{Connections}
The connect operator, :=, is used for connecting two circuit components:
\begin{stanza}
x := y
\end{stanza}
Here is an example of connecting two single wires:
\begin{stanza}
wire x: UInt<1>
wire y: UInt<1>
x := y
\end{stanza}
\end{frame}

\begin{frame}[fragile]{Bulk Connections}
Note that the same operator is used both for connecting single wires as well as bundle and vector types.  Here is an example of connecting two vector types:
\begin{stanza}  
wire x: UInt<1>[10]
wire y: UInt<1>[10]
x := y
\end{stanza}
Here is an example of connecting two user-defined bundle types:
\begin{stanza}  
wire x: Cplx
wire y: Cplx
x := y
\end{stanza}
Notice that the actual connect statement is the same for all three above examples. The only difference is the type declaration of the wires.
\end{frame}

\begin{frame}[fragile]{Literals}
Literals representing unsigned integers are specified like so. The following creates an unsigned integer with value 42, with its bitwidth left unspecified:
\begin{stanza}
UInt(42)
\end{stanza}  
To explicitly give the literal a bitwidth, we may put it in angle brackets:
\begin{stanza}
UInt<16>(42)  
\end{stanza}  
Signed integer and boolean literals are similarly specified:
\begin{stanza}
SInt(-30)
SInt<16>(-30)  
Bool(true)
Bool(false)
\end{stanza}
\end{frame}

\begin{frame}[fragile]{When Statements}
The when statement is used to conditionally connect components. The following code connects x to 0 when enable is asserted otherwise connects x to 10:
\begin{stanza}
wire x: UInt
when enable :
   x := UInt(0)
else :
   x := UInt(10)
\end{stanza}
The else branch may be left off if there are no statements:
\begin{stanza}
wire x: UInt
when enable :
   x := UInt(0)
\end{stanza}
\end{frame}

\begin{frame}[fragile]{Nested When Statements}
If the else branch in a when statement is another single when statement, Chipper allows the colon after the else to be elided for convenience.
  
Thus the following code:
\begin{stanza}
when e :
   x := a
else :
   when f :
      x := b
\end{stanza}
\noindent
is equivalent to:
\begin{stanza}
when e :
   x := a
else when f :
   x := b
\end{stanza}
\end{frame}

\begin{frame}[fragile]{Bits and BitRanges within an Integer}
To extract a bit or a bitrange in an unsigned integer, we may use the following syntax:
\begin{stanza}
wire w: UInt<16>
x := w[0]
\end{stanza}
\noindent
The above example connects x to the least significant bit in w. To extract a bitrange, we give two arguments:
\begin{stanza}
wire w: UInt<16>
x := w[3, 0]
\end{stanza}
As is tradition in hardware, the higher bit is specified first, and the lower bit specified second. The above example connects x to bits 3 (inclusive) through 0 (inclusive) in the wire w.
\end{frame}

\begin{frame}[fragile]{Accessing Vector Elements}
To access a single component within a vector, we may use the following syntax:
\begin{stanza}
wire w: UInt<16>[10]
x := w[4]
\end{stanza}
\noindent
The above code connects x to the fifth component in w.

If we index into a vector type with an expression instead of an integer then an accessor is created to dynamically lookup the value at runtime:
\begin{stanza}
input i: UInt
wire w: UInt<16>[10]
x := w[i]
\end{stanza}
\end{frame}

\begin{frame}[fragile]{Remaining Topics}
\begin{itemize}
\item firrtl transformation passes
\item stanza overview  
\item vec of modules
\item riscv-mini
\item stdlib
\item queues
\end{itemize}
\end{frame}

\end{document}
